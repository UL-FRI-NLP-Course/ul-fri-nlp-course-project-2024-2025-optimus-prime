%%%%%%%%%%%%%%%%%%%%%%%%%%%%%%%%%%%%%%%%%
% FRI Data Science_report LaTeX Template
% Version 1.0 (28/1/2020)
% 
% Jure Demšar (jure.demsar@fri.uni-lj.si)
%
% Based on MicromouseSymp article template by:
% Mathias Legrand (legrand.mathias@gmail.com) 
% With extensive modifications by:
% Antonio Valente (antonio.luis.valente@gmail.com)
%
% License:
% CC BY-NC-SA 3.0 (http://creativecommons.org/licenses/by-nc-sa/3.0/)
%
%%%%%%%%%%%%%%%%%%%%%%%%%%%%%%%%%%%%%%%%%


%----------------------------------------------------------------------------------------
%	PACKAGES AND OTHER DOCUMENT CONFIGURATIONS
%----------------------------------------------------------------------------------------
\documentclass[fleqn,moreauthors,10pt]{ds_report}
\usepackage[english]{babel}

\graphicspath{{fig/}}




%----------------------------------------------------------------------------------------
%	ARTICLE INFORMATION
%----------------------------------------------------------------------------------------

% Header
\JournalInfo{FRI Natural language processing course 2025}

% Interim or final report
\Archive{Project report} 
%\Archive{Final report} 

% Article title
\PaperTitle{Conversational Agent with Retrieval-Augmented Generation for research assistance} 

% Authors (student competitors) and their info
\Authors{Marko Medved, Sebastijan Trojer, and Matej Vrečar}

% Advisors
\affiliation{\textit{Advisors: Aleš Žagar}}

% Keywords
\Keywords{Keyword1, Keyword2, Keyword3 ...}
\newcommand{\keywordname}{Keywords}


%----------------------------------------------------------------------------------------
%	ABSTRACT
%----------------------------------------------------------------------------------------

\Abstract{
The abstract goes here. The abstract goes here. The abstract goes here. The abstract goes here. The abstract goes here. The abstract goes here. The abstract goes here. The abstract goes here. The abstract goes here. The abstract goes here. The abstract goes here. The abstract goes here. The abstract goes here. The abstract goes here. The abstract goes here. The abstract goes here. The abstract goes here. The abstract goes here. The abstract goes here. The abstract goes here. The abstract goes here. The abstract goes here. The abstract goes here. The abstract goes here. The abstract goes here. The abstract goes here.
}

%----------------------------------------------------------------------------------------

\begin{document}

% Makes all text pages the same height
\flushbottom 

% Print the title and abstract box
\maketitle 

% Removes page numbering from the first page
\thispagestyle{empty} 

%----------------------------------------------------------------------------------------
%	ARTICLE CONTENTS
%----------------------------------------------------------------------------------------

\section*{Introduction}
We are developing a conversational agent with Retrieval-Augmented Generation (RAG) to support academic research, focusing on the field of automatic re-identification. The system will leverage a pre-trained chatbot developed in [insert year], enhanced with a RAG mechanism to retrieve and incorporate the latest research papers from arXiv. To keep the model manageable and focused, we will limit the scope of retrieved papers to a specific recent time frame (e.g., one month). The retrieved papers will be processed and integrated into the chatbot's responses, allowing the agent to provide up-to-date and accurate answers about re-identification methods developed after the chatbot’s original training. To evaluate our approach an option is to manually check if the answers were relevant for the chosen scientific area and if the papers used for the queries were relevant for the specific query. Another option would be to compare the answers of a chat-bot that was trained with more recent data. Possible pre-trained chatbots that we could use include Mistral 7B or Zephyr. 





%------------------------------------------------

\section*{Related work}

% Joined

Large Language Models (LLMs) have demonstrated strong language understanding and generation capabilities, but they often struggle with hallucinations, outdated knowledge, and limited domain-specific accuracy, especially when handling specialized research queries \cite{kandpal2023longtail}. Retrieval-Augmented Generation (RAG) addresses these challenges by enhancing LLMs with real-time information retrieval, improving factual accuracy and relevance \cite{gao2024rag}. Early RAG models followed a basic retrieve-then-generate structure, but more advanced approaches have introduced techniques like query rewriting, reranking, and adaptive retrieval to improve context relevance and coherence \cite{gao2024rag}.

Recent research has shown the effectiveness of RAG-based systems for academic and domain-specific applications. For example, \cite{Chatbots_in_Academia} implemented a RAG system using OpenAI Ada embeddings and GPT-3.5 to enhance information retrieval and synthesis for academic documents, demonstrating the importance of precise embedding strategies for research-based chatbots. Afzal et al. \cite{afzal2024towards} focused on domain-specific data by creating the CurriculumQA dataset and optimizing retrieval with tailored evaluation metrics such as relevance, coherence, and faithfulness, highlighting the importance of fine-tuning for specific academic fields. Similarly, Yasin et al. \cite{yasin2024retrieval} integrated advanced techniques—including fine-tuned embedding models, semantic chunking, and abstract-first retrieval—to improve the accuracy and relevance of scholarly content retrieval.

Evaluating RAG systems remains complex due to the nuanced nature of retrieval and generation quality. Chen et al. \cite{chen2024benchmarking} introduced the Retrieval-Augmented Generation Benchmark (RGB) to measure RAG performance across four key dimensions: noise robustness, negative rejection, information integration, and counterfactual robustness. Their findings show that while RAG improves LLM accuracy, it still struggles with integrating long-distance information and handling noisy or uncertain evidence. Common evaluation methods for RAG systems include retrieval-focused metrics like MRR (Mean Reciprocal Rank) and nDCG (Normalized Discounted Cumulative Gain), as well as generation-focused metrics like BLEU and ROUGE to measure the quality and relevance of generated responses \cite{lewis2020rag}.

In our project, we aim to address these issues by developing a RAG-based chatbot for academic research in automatic re-identification—the task of matching individuals across different cameras or settings, a complex problem in computer vision and machine learning. By focusing on recent research papers from arXiv, we seek to enhance the chatbot’s ability to retrieve and generate accurate responses about the latest developments in automatic re-identification


\section*{Methods}
\subsection*{Basline}
Our baseline method for retrieving and responding to academic
 queries begins with collecting relevant scientific content using the 
 Python arxiv module. This allows us to retrieve titles, abstracts, and author
  information for a large number of recent papers from arXiv. To enable semantic
   matching between user queries and the retrieved documents, we employ the allenai-specter
    model, a transformer-based embedding model specifically trained for scientific papers.
     Both the user query and the paper abstracts are embedded into a shared vector space, 
     and cosine similarity is computed to identify the top five most relevant papers. 
     These selected paper summaries form the context for response generation. We use 
     the langchain module to pass this context to our base conversational agent, which
      is powered by the Mistral-7B-Instruct-v0.2 model, quantized to 4 bits to ensure 
      efficient inference. This pipeline provides a lightweight
       retrieval-augmented generation (RAG) setup tailored to scientific literature.

\section*{Results}
\subsection*{Preliminary results }
To evaluate the effectiveness of our RAG-enhanced approach, we tested it on scientific
 questions related to papers published after December 2023. The goal was to determine 
 whether the system could incorporate recent research not seen during the original
  model’s training.

Across a range of prompts, results showed that while the system often failed to retrieve 
the exact target paper with top-$k=5$, it still produced more informative and current responses by leveraging similar recent articles. Increasing $k$ to 10 improved retrieval in several cases—for example, retrieving a nearly identical sentiment analysis paper by the same authors.

Refined queries also led to improved results. For a fast and robust covariance estimator, adding terms like “low contamination regime” and “near-linear sample complexity” helped surface the correct article. In contrast, generic queries often led the model to irrelevant domains, such as quantum mechanics.

For topics like NP-complete problems and object detection, specificity again proved critical. While exact matches were rare, the RAG system offered responses more reflective of recent research trends than the baseline. However, it occasionally exhibited erratic behavior, such as generating its own sub-questions, which significantly increased inference time.

Overall, these preliminary findings, detailed further in \texttt{results/testing\_the\_model.docx},
 suggest that RAG enhances response relevance and currency, particularly with higher $k$
  and well-formed queries. However, retrieval control and relevance filtering remain
  areas for future improvement. Note that these results were collected using the baseline method. 
  



%------------------------------------------------


%------------------------------------------------

\section*{Discussion}

Use the Discussion section to objectively evaluate your work, do not just put praise on everything you did, be critical and exposes flaws and weaknesses of your solution. You can also explain what you would do differently if you would be able to start again and what upgrades could be done on the project in the future.


%------------------------------------------------

\section*{Acknowledgments}

Here you can thank other persons (advisors, colleagues ...) that contributed to the successful completion of your project.


%----------------------------------------------------------------------------------------
%	REFERENCE LIST
%----------------------------------------------------------------------------------------
\bibliographystyle{unsrt}
\bibliography{report}


\end{document}